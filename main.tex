\documentclass[11pt]{report}
\usepackage{styles} % the file containing styles for definitions, propositions, proofs etc.
\usepackage{subfiles}
%\usepackage{amsfonts}
\usepackage{tcolorbox}



\title{\textbf{PMATH 340: Elementary Number Theory}}
\author{Notes Taken By:\\Evan Bernard\\\\Class Taught By:\\Professor Wentang Kuo}
\date{Winter 2020\\ University of Waterloo}


\begin{document}
\maketitle
\tableofcontents

\begin{abstract}
Number Theory is simply the study of integers. This course analyzes some interesting relationships between integers, and the implications of these relationships. The course can be thought of as being split into three main sections, the first being solving \nmod{x^2}{a}{p}, the second $x^2+y^2=n$, and finally, $x^2-Dy^2=1$. You are expected to be familiar with what was taught in MATH 135, including but not limited to, \textit{linear Diophantine equations, euclidean algorithm, congruence} and the \textit{Chinese remainder theorem.}
\end{abstract}

% Pythagorean Triplets (integer solutions to c^2=a^2+b^2)
\subfile{Chapters/Chapter01}
% Euler's Formula
\subfile{Chapters/Chapter02}
% Congruence Relations
\subfile{Chapters/Chapter03}
% Squares modulo p
\subfile{Chapters/Chapter04}
% Integer points on a circle
\subfile{Chapters/Chapter05}
% Integer solutions to x^2-dy^2=1
\subfile{Chapters/Chapter06}

\end{document}
