\documentclass[main.tex]{subfiles}

\begin{document}
\chapter{idk what to call this chapter yet}

\section{Integral Pythagorean Triplets}
Question: Find all integral Pythagorean triples.

In other words, find all integer triples \(a, b, c\), such that \(a, b, c\) satisfy the Pythagorean formula,
\(c^2 = a^2 + b^2\). We may assume that \(a, b > 0\) since if \(a = 0\) or \(b = 0\) then the solutions are known.

Find all triplets \(a, b, c\) such that $a, b, c \in $ $\N$ and $c^2 = a^2 + b^2$

\paragraph{Examples:}
\begin{center}
    $5^2 = 3^2 + 4^2$ \\
    $13^2 = 5^2 + 12^2$ \\
    $17^2 = 8^2 + 15^2$ \\
    $25^2 = 7^2 + 24^2$
\end{center}
Note that, given a Pythagorean triple $a, b, c$, and an integer $d$, then $da, db, dc$ form another Pythagorean triplet.

\begin{defn}
    A Primitive Pythagorean triplet, or $PPT$, is a triple of natural numbers $a, b, c$ so that $a, b, c \in \N$ and $gcd(a, b, c) = 1$ since otherwise, you can divide by a common factor and find a smaller triplet.
\end{defn}

We can now rewrite the question into a simpler question, which reduces the question into the essential component of it: Find all primitive Pythagorean triples.

\begin{prop}
    $c$ is odd.
\end{prop}

\begin{pf}
Suppose $c$ is even.
Then $c=2k$ for $k \in \N$


\underline{Case 1:} both $a$ and $b$ are even. \\
Then $gcd(a, b, c) = 2 \neq 1$ thus $a, b, c$ are not $PPT$

\underline{Case 2:} both $a$ and $b$ are odd. \\
Then there exists $x, y \in \Z$ such that $a = 2x + 1$ and $b = 2y + 1$, and since $c^2 = a^2 + b^2$, we have that:

\begin{center}
    $c^2 = (2x + 1)^2 + (2y + 1)^2$ \\
    $4z^2 = 4(x^2 + x y^2 + y) + 2$ \\
    $4(z^2 - x^2 - x - y^2 - y) = 2$ \\
    $2(z^2 - x^2 - x - y^2 - y) = 1$
\end{center}

This is a contradiction, and thus, our first assumption is invalid. Therefore $c$ must be an odd integer.
\end{pf}

So far, we know that if $a, b, c$ is a $PPT$, then $c$ is odd and without loss of generality, we may assume that $a$ is even and $b$ is odd.

By rearranging the Pythagorean formula, we get:

\centerline{$a = \sqrt{(c-b)(c+b)}$}

\begin{prop}
    If $a$ is odd, then $(c-b)(c+b)$ are squares.
\end{prop}

\begin{pf}
    Let $p$ be a prime divisor of $a$. Then, 
    \begin{center}
        $p^2 \mid a^2$ \\
        $p^2 \mid (c + b)(c - b)$
    \end{center}

    Suppose $p$ divides one of $(c + b)$ and $(c - b)$ but not both, then $(c + b)$ and $(c - b)$ are squares by the \textit{Unique Factorization Theorem.} \\ 
    Therefore it is enough to show that $gcd(c + b, c - b) = 1$.
    
    Let $d = gcd(c + b, c - b)$, then we have the following:
    \begin{center}
        $d \mid c + b$ \\
        $d \mid c - b$ \\
        $\implies d \mid (c + b) + (c - b) = 2c$ \\
        $\implies d \mid (c + b) - (c - b) = 2b$ \\
    \end{center}
\end{pf}
\begin{pf}[Cont.]
    Thus, we have that,
    \begin{center}
        $d \mid gcd(2b, 2c)$ \\
        $d \mid 2gcd(b, c)$ \\
        $d = 1$ or $d = 2$
    \end{center}
    Suppose $d = 2$, we know that,
    \begin{center}
        $d \mid (c-b)$ and $d \mid (c-b)(c+b) = a^2$ \\
        $\implies 2 \mid a^2$ \\
        $\implies 2 \mid a$ \\
        $\implies$ a is even
    \end{center}
    But we assumed that a was odd, so $d$ must be equal to 1. Since $d=gcd(c+b,c-b)$ and $d=1$, $(c-b)$ and $(c+b)$ are coprime, and so are squares by the\textit{Unique Factorization Theorem}.  
\end{pf}
Since we have that $(c-b)$ and $(c+b)$ are coprime and squares, we also have that $\exists s, t \in \N$ where $gcd(s,t)=1$ and $c-b = t^2$ and $c+b = s^2$
\begin{center}
    $\implies c = \dfrac{s^2+t^2}{2}$ \\
    $b = \dfrac{s^2-t^2}{2}$ \\
    $a^2 = c^2 - b^2 = s^2 \cdot t^2$ \\
    $\implies a = s \cdot t$
\end{center}
\begin{thm}[Pythagorean Triple Theorem]
    Every primitive Pythagorean triple $(a,b,c)$, where $a,c$ are odd, $b$ is even, can be obtained by $a=s \cdot t$, $b=\dfrac{s^2 - t^2}{2}$, $c=\dfrac{s^2 + t^2}{2}$, where $s,t \in \N, s > t$, and $s,t$ are odd and coprime $(gcd(s,t)=1)$.
\end{thm}
\paragraph{Example: $s = 3$, $t = 1$}
\begin{center}
    $b=\dfrac{3^2-1^2}{2} = 4$ \\
    $c=\dfrac{3^2+1^2}{2} = 5$ \\
    $a=3 \cdot 1 = 3$ \\
    A $PPT$ is $(3, 4, 5)$
\end{center}
It's Important to note that this does not guarantee the result to be a $PPT$ for any $s$ and $t$ satisfying the conditions, however, every $PPT$ will have an $s$ and $t$ decomposition.

\section{Euler's Formula}
\begin{defn}[Congruence]
    Let $a,b,m \in \Z, m \in \N$. We say that $a$ is \textit{congruent} to $b$ modulo $m$, if $m \mid (b-a)$. We use the following notation: \\
    \centerline{$a \equiv b$  $(mod$ $m)$}
\end{defn}
\begin{thm}
    If $a,b,c \in \Z, m \in \N$ and $gcd(c,m)=1$, then, \\
    \centerline{$a \cdot c \equiv b \cdot c$ $(mod$ $ m)$} \\
    \centerline{$\implies a \equiv b$ $(mod$ $ m)$}
\end{thm}
\paragraph{Question:}
Given $a \in \Z, m \in \N,$ find an integer $\gamma \in \N$, such that $a^\gamma \equiv 1$ $(mod$ $m)$
\begin{thm}[Fermat's Little Theorem \textit{(FlT)}]
    $\forall a, p \in \Z,$ prime $p$, $gcd(a,p)=1$ then, \\
    \centerline{\nmod{a^{p-1}}{1}{p}}
\end{thm}
So part of the question is trivial with $FlT$, namely, when $m$ is prime then we can simply let $\gamma$ be $m-1$ and we've found a solution to the problem.

Finding an integer $\gamma$ satisfying the equation for any integer $m$ however, is much more difficult.
\begin{defn}[Reduced Residue Class Set]
    Let $m \in \N$, we define the reduced residue class set as: \\
    \centerline{$R_m = \{b \in \Z$: $1 \leq b \leq m$, $gcd(b,m)=1\}$}
\end{defn}
\paragraph{Example:}
\begin{center}
    The residue class set $R_p$ for a prime number $p$ is the following: \\
    $R_p = \{1,2,3,4, ... , p-1\}$
\end{center}
Recall that the key step for proving $FlT$ is to see that: \\
\centerline{\nmod{\{1,2,3,...,p-1\}}{\{a,2a,3a,...(p-1)a\}}{p} for $p \nmid a$.}

\paragraph{Example: $a=3$, $p=5$}
\begin{center}
    $R_5=\{1,2,3,4\}$ \\
    abusing notation, \\
    $3 \cdot R_5 = \{3,6,9,12\}$ \\
    and notice that when we take the mod 5 of each element, we get, \\
    $3 \cdot R_5$ $mod$ $5$ $\equiv \{3,1,4,2\}$ \\
    which is exactly $R_5$ in a different order
\end{center}

\begin{defn}[Euler's Phi (Totient) Function]
    Let $m \in \N$. Define $\phi(m):=$ \# elements in $R_m$ \\
    \centerline{$\phi(m)=$ \# elements in \residue{b}{Z}{m}{b}}
\end{defn}

\paragraph{Example 1: $\phi(p)=p-1$ \\}
Since $R_p=\{1,2,3,...p-1\}$, there are $p-1$ elements in $R_p$

\paragraph{Example 2: $\phi(p^k)=p^k-p^{k-1}$ \\}
Constructing $R_{p^k}$, it's clear that there would be $p^k$ elements in the set if we were to include the values of $b$ which don't satisfy the condition that $gcd(b,p^k)=1$. The question now is how many values of $b$ are there which don't satisfy the condition? The only time $gcd(b,p^k) \neq 1$ is when $b \mid p^k$. Since $p$ is prime, this is also when $b$ is a multiple of $p$. There are exactly $p^{k-1}$ multiples of $p$, and so the value of Euler's Phi Function is $p^k - p^{k-1}$.

\begin{thm}[Euler's Formula]
    Let $a \in \Z$, $m \in \N$, $gcd(a,m)=1$. Then, \\
    \centerline{\nmod{a^{\phi(m)}}{1}{m}.}
\end{thm}
\textbf{Remark:} By Euler's formula, we can see that when $m=p$ is prime then \nmod{a^{p-1}}{1}{p}, so $FlT$ is just a special case of Euler's formula.
\begin{lem}
    \nmod{R_m}{a \cdot R_m}{m}
\end{lem}
\begin{pf}[Lemma 1.2.0.4]
    We will do this proof on friday, for now we will assume it's true to prove Euler's formula.
\end{pf}
\begin{pf}[Euler's Formula]
    By Lemma 1.2.0.4, we have that, \\
    \begin{center}
        \nmod{R_m}{a \cdot R_m}{m} \\
        Suppose we have that, \\
        $R_m=$ \residue{b}{Z}{m}{b} \\
        $a \cdot R_m=$ \residue{a \cdot b}{Z}{m}{b} \\
    \end{center}
    Now consider what we get if we take the product of all terms in these sets. Notice that,
    $$\prod_{b \in R_m} b \equiv \prod_{b \in R_m} a \cdot b \mod{m}$$ \\
    Thus we have that,
    $$\prod_{b \in R_m} b \equiv a^{\phi(m)} \cdot \prod_{b \in R_m} b \mod{m}$$ \\
    and since $\forall$ $b \in R_m$, $gcd(b,m)=1$, we're able to cancel out the cartesian products and are left with the following:
    \begin{center}
        \nmod{1}{a^{\phi(m)}}{m}
    \end{center}
\end{pf}

\end{document}
