\documentclass[../main.tex]{subfiles}

\begin{document}

\chapter{Congruence Relations}

\section{Successive Squaring algorithm}
\paragraph{Question:}
Compute the remainder $r$ of when $9^{53}$ is divided by $67$. In other words, solve for $r$ in \nmod{9^{53}}{r}{67}.
\paragraph{Motivation:}
Using theorems we know, such as the \textit{congruence power theorem} from MATH135, we are required to use $53$ as an exponent, which is far too large for us to compute. Thus, a better method is needed for computing these remainder problems for large exponents.
\begin{thm}[Method of Successive Squaring To Compute $a^k$ $mod$ $m$]
    \underline{Step 1:} Write $k$ as a sum of powers of 2 \\
    \underline{Step 2:} Make a table of the powers of a modulo $m$ using successive squaring \\
    Then, $$a^k \equiv \prod_{i=1}^{t} (A_i)^{u_i} \mod{m}$$
\end{thm}

Now that we have the resources to answer the question, lets go through the question.
\paragraph{Question:}
Compute the remainder $r$ of when $9^{53}$ is divided by $67$. In other words, solve for $r$ in \nmod{9^{53}}{r}{67}.

\begin{ans}
    \underline{Step 1:} (Write the exponent as a sum of powers of 2)
    \begin{center}
        $53 = 2^5 + 21$ \\
        $53 = 2^5 + 2^4 + 2^2 + 2^0$
    \end{center}
    The largest exponent is $5$ in the rewritten number, and so for the next step, we will go up to $9^{2^5} = 9^{32}$ \sspace
    \underline{Step 2:} (Create a table of the powers of $9$ modulo $67$)
    \begin{center}
        \begin{tabular}{c c c c}
            $9^1\equiv$&9& &$\equiv 9 (mod$ $67)$ \\
            $9^2\equiv$&81& &$\equiv 14 (mod$ $67)$ \\
            $9^4\equiv$&$(14^2) \equiv$&196&$\equiv 62 (mod$ $67)$ \\
            $9^8\equiv$&$(62^2) \equiv$&3844&$\equiv 25 (mod$ $67)$ \\
            $9^{16}\equiv$&$(25^2) \equiv$&625&$\equiv 22 (mod$ $67)$ \\
            $9^{32}\equiv$&$(22^2) \equiv$&484&$\equiv 15 (mod$ $67)$ \\
        \end{tabular}
    \end{center}
    Notice that, 
    \begin{center}
        $9^{53} \equiv 9^{32} \cdot 9^{16} \cdot 9^{4} \cdot 9^{1}$ $(mod$ $67)$ \\
        $9^{53} \equiv 15 \cdot 22 \cdot 62 \cdot 9$ $(mod$ $67)$ \\
        $9^{53} \equiv 24$ $(mod$ $67)$
    \end{center}
    Thus we have that the remainder when $9^{53}$ is divided by $67$, is $24$. \qed
\end{ans}


\section{Bases in Congruence Relations}
\paragraph{Question:}
Let $k,m \in \N$, $b \in \Z$. Solve the congruence relation for $x$: \\
\centerline{$x^k \equiv b$ $(mod$ $m)$}
Note that so far we know how to solve for $k$ if we are given $x,b$ and $m$, but now we are looking to solve for $x$ when given $k,b$ and $m$. \sspace
We can very easily solve it if $k=1$, so lets suppose $k>1$.
\begin{thm}[Solving For x in Congruence Relations]
    Let $b,k,m \in \Z$, $k \geq 1$, $m \geq 1$ such that we have that $gcd(b,m)=1$ and $gcd(k, \phi(m))=1$. Then, the following steps will give a solution to \\
    \centerline{\nmod{x^k}{b}{m}}
    \underline{Step 1:} Compute $\phi(m)$ \sspace
    \underline{Step 2:} Find positive integers $u,v \in \N$ that satisfy $k \cdot u - \phi(m) \cdot v = 1$. Note: this step is from the linear diophantine equation theorem, which states that $a\cdot x + b \cdot y = c$ has a solution if $gcd(a,b) \mid c$. We can find solutions to this using the \textit{Euclidean Algorithm} \sspace
    \underline{Step 3:} $x=b^u$ $(mod$ $m)$ is a solution to the relation
\end{thm}

\begin{pf}
    It's enough to show that $x=b^u$ is a solution.
    \begin{center}
        $(b^u)^k = b^{ku} = b^{1+ \phi(m)v} = (b^{\phi(m)})^vb$
    \end{center}
    By Euler's formula, \nmod{b^{\phi(m)}}{1}{m}, so, \\
    \centerline{$(b^u)^k \equiv 1^vb \equiv b$ $(mod$ $m)$}
    So, $x=b^u$ is a solution to the equation. \qed
\end{pf}
\paragraph{Remark:} This theorem gives us a way to find the $k^{th}$ root of $b$ modulus $m$.

\paragraph{Example:} Solve \nmod{x^{25}}{7}{135}.
\begin{ans}
    Notice that $gcd(7,135)=1$ and $gcd(25, \phi(135))=1$ \\
    The conditions for our theorem are met, so we may use it. \sspace
    We have the following linear diophantine equation: \\
    \centerline{$25 \cdot u + 72 \cdot v = 1$}
    Solved by \textit{E.Z.A}, gives $u=49, v=17$. \sspace
    By our theorem, we have that $x=7^{49}$ is a solution. We can perform $7^{49}$ $(mod$ $135)$ to find a smaller solution to our equation using successive squaring. Since an example of successive squaring is given on the previous page, this will be skipped. The answer after successive squaring is $52$ (this is different than the answer given in class (88), but this is the correct answer), and so we know that \nmod{52^{25}}{7}{135}, as required.
\end{ans}

\paragraph{Example:} Solve \nmod{x^4}{7}{15}
\begin{ans}
    Notice that $gcd(7,15)=1$ and $gcd(4, \phi(15))=4 \neq 1$, so we cannot use our theorem. \sspace
    We can, however, use the \textit{Splitting Module Theorem} from MATH 135. \\
    By $SMT$, \nmod{x^4}{7}{15} if and only if \nmod{x^4}{7}{5} and \nmod{x^4}{7}{3}. \sspace
    Consider \nmod{x^4}{7}{5}. \\
    Notice that if $x$ is a solution, it must be coprime to $5$, since $gcd(5,7)=1$. So, by $FlT$, we have that \nmod{x^4}{1}{5}, however, \nmod{7}{2}{5}, which is a contradiction, and so \nmod{x^4}{7}{5} has no solution. \\
    Thus, \nmod{x^4}{7}{15} has no solution by $SMT$.
\end{ans}

\section{Sums of Divisors with Phi}
\paragraph{Question:} Let $n \in \N$, and $d_1,d_2,...,d_k$ be all positive divisors of $n$. What is the sum of $\phi(d_1) + \phi(d_2) + ... + \phi(d_k)$?
\begin{ans}
    The sum is equal to $n$. This will be proven as a theorem, but for now lets consider the summation when $n$ is prime. \sspace
    Let $n=p$, where $p$ is prime. \\
    $d_1=1$, $d_2=p$ are the only divisors by the properties of primes. Thus we have, \\
    \centerline{$\phi(1) + \phi(p) = 1 + p-1 = p=n$}
    
    Let $n=p^k$, where $p$ is prime. \\
    $d_1=1$, $d_2=p$, $d_3=p^2$, $...$, $d_k=p^k$ are the only divisors of n.
    \begin{center}
        $\phi(1)+\phi(p)+\phi(p^2)+...+\phi(p^k)$ \\
        $=1+(p-1)+(p^2-p)+...+\phi(p^k-p^{k-1})$ \\
        Notice the cancellation, \\
        $=p^k=n$
    \end{center}
    We've shown that when $n$ is prime or has a prime base, the sum of Euler's phi function of the divisors of $n$ is equal to $n$. This will be helpful since every integer can be expressed as a product of primes.
\end{ans}

\begin{thm}[Euler's Phi Function Summation Formula]
    For $n \in \N$ and $d \geq 1$,
    $$\sum_{d \mid n}\phi(d)=n$$
\end{thm}
\begin{pf}
    Let $F=\sum_{d \mid n}\phi(d)$. \sspace
    If we can show for $m,n \in \N$, if $gcd(m,n)=1$, then $F(m\cdot n)=F(m)\cdot F(n)$, then we are done, since we can express any number as a product of primes, and we know that $F(p^K)=p^k$ for any prime $p$. \sspace
    Let $m=p_1^{d_1}p_2^{d_2}...p_k^{d_k}$, and let $n=\gamma_1^{t_1}\gamma_2^{t_2}...\gamma_j^{t_j}$ \\
    Notice how since $gcd(m,n)=1$, all $p_k$ are different from $\gamma_j$. By the \textit{Divisors From Prime Factorization Theorem}, any divisor $d$ of $m \cdot n$ has the form of a subset of the product of primes multiplied together, i.e, if we let $e=p_1^{\omega_1}p_2^{\omega_2}...p_l^{\omega_l}$ and $f=\gamma_1^{i_1}\gamma_2^{i_2}...\gamma_w^{i_w}$, where $e \mid m$ and $f \mid n$, then $d=e \cdot f$. \sspace
    Thus, any divisor $d$ of $m \cdot n$ is a product of $e$ (a positive divisor of $m$) and $f$ (a positive divisor of $n$), where $gcd(e,f)=1$ since $gcd(m,n)=1$.
\end{pf}
\begin{pf}[cont.]
     The converse of this is also true. If $e$ is a positive divisor of $m$, and $f$ is a positive divisor of $n$, then $e\cdot f$ is a positive divisor of $m \cdot n$, with $gcd(e,f)=1$. \sspace
    To sum up what we have so far, let $\{e_1,e_2,...,e_s\}$ be the set of all positive divisors of $m$, and let $\{f_1,f_2,...,f_u\}$ be the set of all positive divisors of $n$. Then, $\{e_i\cdot f_j:1\leq i \leq s, 1 \leq j \leq u\}$ is the set of all positive divisors of $m\cdot n$. Notice that all $gcd(e_i,f_j)$ are coprime by definition.
    $$F(m\cdot n)=\sum_{i=1}^s \sum_{f=1}^u \phi(e_i \cdot f_j)$$
    $$=\sum_{i=1}^s \sum_{f=1}^u \phi(e_i) \cdot \phi(f_j$$
    $$=\sum_{i=1}^s \phi(e_i) \cdot \sum_{f=1}^u \phi(f_j)$$
    $$=F(m)\cdot F(n)$$ \qed
\end{pf}

\section{Primitive Roots Modulo p}
\paragraph{Question:}
Let $m \in \N$. For some $a \in Z$, where $gcd(a,m)=1$, what is the smallest positive integer $k$ such that \nmod{a^k}{1}{m}? \sspace
We know by Euler that $k=\phi(m)$ is a solution to the equation, so we can say that the smallest integer $k$ always exists, and $k \leq \phi(m)$.

\begin{defn}[Exponent of $a$ modulo $m$]
    Let $a \in \Z$, $m \in \N$, where $gcd(a,m)=1$. The smallest exponent $e$ such that \nmod{a^e}{1}{m} is called the exponent of $a$ modulo $m$, and is denoted by $e_m(a)$. Note that $e_m(a) \leq \phi(m)$.
\end{defn}

\paragraph{Example:} Compute $e_7(5)$. i.e., find the smallest integer $e$ such that \nmod{5^e}{1}{7}.
\begin{ans}
    First, compute $\phi(7)=6$, so the possible values of $e_7(5)$ are between 1 and 6.
    \begin{center}
        $5^1 \equiv 5 \not\equiv 1$ $(mod$ 7) \\
        $5^2 \equiv 25 \equiv 4$ $(mod$ 7) \\
        $5^3 \equiv 4 \cdot 5 \equiv 6$ $(mod$ 7) \\
        $5^4 \equiv 6\cdot 5 \equiv 2$ $(mod$ 7) \\
        $5^5 \equiv 2\cdot5 \equiv 3$ $(mod$ 7) \\
        $5^6 \equiv 3\cdot5 \equiv 1$ $(mod$ 7)
    \end{center}
    So, the smallest exponent $e$ which solves the congruence relation \nmod{5^e}{1}{7} is the maximum value it could have been, i.e, equal to $\phi(7)=6$.
\end{ans}
To give some motivation for the next proposition, notice how long it would take to calculate $e_m(a)$ for large values of $\phi(m)$. Surely, not all positive integers between 1 and $\phi(m)$ are possible values for $e$, so let's try to reduce the values from this range even further.
\begin{prop}[Exponent Divisibility Property]
    Let $a,m \in \Z$, with $gcd(a,m)=1$, and suppose there is a solution to the equation \nmod{a^n}{1}{m} for some $n\in \N$. Then, we have that $$e_m(a) \mid n$$ In particular, $$e_m(a) \mid \phi(m)$$
\end{prop}
\paragraph{Remark:} In our last example in which we calculated $e_7(5)$, we had to calculate the value of $5^n$ modulo 7 for all $1\leq n\leq \phi(7)$. If we can prove this proposition, we would have only needed to calculate the value of $5^n$ modulo 7 for the divisors of $\phi(7)$, so 1,2 and 3.

\begin{pf}[Exponent Divisibility Property]
    By definition, $a^{e_m(a)} \equiv 1$ $(mod$ m). Let $d = gcd(e_m(a), n)=e_m(a)$, since $e_m(a)$ is primitive. \sspace
    By the \textit{Linear Diophantine Equation Theorem}, there exists two positive integers $u,v$ such that $e_m(a)\cdot u - n \cdot v = d$. Then, $a^{e_m(a)\cdot u}=(a^{e_m(a)})^u \equiv 1^u \equiv 1$ ($mod$ m) $$1 \equiv a^{e_m(a)\cdot u} = a^{n\cdot u + d} = (a^n)^u \cdot a^d \equiv 1^u \cdot a^d \equiv a^d \text{ (}mod \text{ m)}$$
    But, by definition, $d=e_m(a)$, and so we have $e_m(a) \mid n$ as required. \qed
\end{pf}
\begin{defn}[Primitive Roots mod m]
    Let $m \in \Z$ and let there exist a positive integer $e_m$ such that, for all $a \in \Z$ where $gcd(a,m)=1$, we have that $a^{e_m} \equiv 1$ $(mod$ $m)$, and $e_m$ is the smallest positive integer satisfying this property. \sspace
    Then, a positive integer $g$ such that $gcd(g, m)=1$ is called the primitive root mod m if $e_m(g)=e_m$
\end{defn}
In particular, if $m=p$ for a prime $p$, $e_p=p-1$. A primitive root $g$ of $p$ satisfies $e_p(g)=p-1$.

\begin{thm}
    For any prime $p$, there exists a primitive root modulo $p$. In other words, for any prime $p$, there exists $g \in \Z$ such that $e_p(g)=p-1=\phi(p)$.
\end{thm}

\paragraph{Remark:} The definition for primitive roots here is different than the one in the textbook, however, they are the same for primes, which is what we will be focusing on in this course. \sspace
We would like to prove this theorem, but we need to define a useful function and lemma first.
\begin{defn}
    Let $p$ be prime and $n$ be a divisor of $\phi(p)=p-1$. We define $$\psi(n):=\#\{a:\text{ }1 \leq a < p,\text{ } e_p(a)=n\}$$In other words, $\psi(n)$ gives the number of values of $a < p$ which satisfies the equation $a^n \equiv 1$ $(mod$ $p)$
\end{defn}

\paragraph{Remark:} The primitive root theorem is equivalent to $\psi(p-1) \geq 1$, so we will prove this instead.

\paragraph{Examples:} Let $p=7$
\begin{center}
    $\psi(1)=\#\{1\}=1$ since \nmod{1^1}{1}{7} is the only solution under 7. \\
    $\psi(2)=\#\{6\}=1$ since \nmod{6^2}{1}{7} is the only solution under 7. \\
    $\psi(3)=\#\{2,4\}=2$ since \nmod{2^3}{1}{7} and \nmod{4^3}{1}{7}
\end{center}

Notice that it seems to be true that $\phi(n)=\psi(n)$. We can easily verify for the examples we've done that this holds, but it's difficult to show it holds for all $n,p$. If it is true, then $\psi(p-1)=\phi(p-1) \geq 1$, proving the theorem is true. We need to introduce a theorem which will help us prove this.

\begin{thm}[Lagrange's Theorem]
    Given a polynomial $f(x)$ of degree $n, n \in \N$, the number of solutions to the congruence relation \nmod{f(x)}{0}{p} for a prime $p$ is at most $n$. 
\end{thm}
\begin{pf}
    Let $x=a_1$, $1 \leq a_1 < p$ be a solution to \nmod{f(x)}{0}{p}. We will be referring to the following congruence relation often, and so let $\star$ represent \nmod{f(x)}{0}{p}. \sspace
    The factor theorem from MATH135 says that for any $c \in F$, we have \\
    \centerline{$(x-c)\mid f(x) \iff f(c)=0$} Where $F$ is a field. In our case, $F=\Z/p\Z$ for prime p, which is a field (this notation is the proper notation, it means the set of integers in modulo p). \sspace
    So, since $a_1$ is a solution, we have that $(x-a_1) \mid f(x)$. If we repeat the process $n$ times, we're left with the following \\
    \centerline{$f(x)\equiv (x-a_1)(x-a_2)...(x-a_n)g_n(x)$}
    Where $a_1,a_2,...,a_n$ are solutions to $\star$. We need to show that there cannot be any more solutions, so lets assume we have another solution $a^0$ to $\star$. Then, we have
    \centerline{$0 \equiv f(a^0)=(a^0-a_1)(a^0-a_2)...(a^0-a_n)g_n(a^o)$ $(mod$ $p)$}
    Since we are in the field $\Z/p\Z$, it is also a domain (if $ab=0$ then $a=0$ or $b=0$). Thus, we have that at least one factor of $f(a^0)$ is equal to $0$, but we defined $a^0$ to be a unique solution, so none of $(a^0-a_n)=0$. This means that $g_n(x)\equiv 0$ $(mod$ $p)$. \sspace
    By the factor theorem,
    \begin{center}
        $(x-a^0)\mid g(a^0)$ \\
        and also, \\
        $f(x)\equiv (x-a_1)(x-a_2)...(x-a_n)(x-a^0)g_n(x)$ $(mod$ $p)$
    \end{center}
    This has degree $n+1$, contradicting our assumption from the theorem that the degree of the polynomial was $n$. Thus, the number of solutions to \nmod{f(x)}{0}{p} is at most $n$.
\end{pf}
\paragraph{Example:} Consider \nmod{x^5}{1}{11} \\
The solutions are, $x=1,3,4,5,9$, so there are 5 solutions, and a degree of 5.

\begin{lem}
    Let $p$ be a prime number. If $n \mid (p-1)$, then the congruence \nmod{x^n-1}{0}{p} has exactly $n$ solutions.
\end{lem}
\begin{pf}
    Let $p-1=nk$. \\
    We have that 
    \begin{center}
        $x^{p-1}-1=x^{nk}-1=(x^n)^k-1$ \\
        $=(x^n-1)((x^n)^{k-1}+(x^n)^{k-2}+...+(x^n)^2+(x^n)^1+1)$
    \end{center}
    By $FlT$, the congruence \nmod{x^{p-1}-1}{0}{p} has $p-1$ solutions, ie. $(1,2,3,...,p-1)$. Since \nmod{x^n-1}{0}{p} has at most $n$ solutions by the previous theorem, and \nmod{(x^n)^{k-1}+(x^n)^{k-2}+...+(x^n)^2+(x^n)^1+1}{0}{p} has at most $n(k-1)$ solutions, \nmod{x^n-1}{0}{p} must have exactly $n$ solutions. \qed
\end{pf}
To sum up what we've done, we've proven the following proposition:
\begin{prop}
    If $n \mid p-1$ and $d_1,d_2,...,d_r=n$ are all the divisors of $n$, then $\psi(d_1)+\psi(d_2)+...+\psi(d_r)=n$.
\end{prop}
Recall that $\phi$ also satisfies the above equation. Now we can finally use these propositions to prove the following theorem:
\begin{thm}
    For all $n \in \Z$, we have that $\psi(n)=\phi(n)$
\end{thm}
\begin{pf}
    Let $p$ be a prime \\
    Clearly, $\phi(1)=1=\psi(1)$. \\
    Let $q$ be a prime factor of $p-1$. Then, by the above proposition, $\psi(1)+\psi(q)=q$. Thus, $\psi(q)=q-\psi(1)=q-\phi(1)=q-1=\phi(q)$. \sspace
    Let $qr$ be a divisor of $p-1$, where $q,r$ are prime. Then, by the above proposition, \\
    \centerline{$\psi(qr)=qr-\psi(q)-\psi(r)-\psi(1)=qr-\phi(q)-\phi(r)-\phi(1)=\phi(qr)$}
    Since we can express every $n$ as the product of primes, we have that $\phi(n)=\psi(n)$ for all $n \in \Z$. \qed
\end{pf}

\begin{defn}[Index of $a$ mod $p$ for base $g$]
    Let $p$ be a prime and $g$ and primitive root of $p$ (ie. \nmod{a^g}{1}{p} for all $a \in \Z$). Let $a$ be a non-zero number modulo $p$. Then, there is a unique number $i$ such that $1 \leq i \leq p-1$ such that \nmod{a}{g^i}{p}. The exponent $i$ is called the index of $a$ modulo $p$ for the base $g$. Assuming that $p$ and $g$ have been specified, we write $I(a)$ for the index.
\end{defn}

\begin{prop}
    Let $g$ be a primitive root of $p$. Then, every non-zero number modulo $p$ is congruent to a power of $g$. More precisely, for any number $1 \leq a < p$, we can pick exactly one of the powers $g, g^2, g^3,...,g^{p-1}$ as being congruent to $a$ modulo $p$.
\end{prop}
\paragraph{Example:} Let $p=13$, and $g=2$. Then, \\

\begin{center}
    \begin{tabular}{ c|c c c c c c c c c c c c } 
        I & 1 & 2 & 3 & 4 & 5 & 6 & 7 & 8 & 9 & 10 & 11 & 12 \\
        \hline
        $2^I$ $(mod$ 13) & 2 & 4 & 8 & 3 & 6 & 12 & 11 & 9 & 5 & 10 & 7 & 1\\ 
    \end{tabular}
\end{center}
Read back the numbers to get the indices.
\begin{center}
    \begin{tabular}{ c|c c c c c c c c c c c c } 
        a & 1 & 2 & 3 & 4 & 5 & 6 & 7 & 8 & 9 & 10 & 11 & 12 \\
        \hline
        $I(a)$ & 12 & 1 & 4 & 2 & 9 & 5 & 11 & 3 & 8 & 10 & 7 & 6\\ 
    \end{tabular}
\end{center}

\begin{thm}[Index Rules Theorem]
    Let $p$ be a prime and $g$ a primitive root of $p$. Then,
    \begin{itemize}
        \item \nmod{I(ab)}{I(a)+I(b)}{p-1} (product rule)
        \item \nmod{I(a^k)}{k \cdot I(a)}{p-1} (power rule)
    \end{itemize}
\end{thm}
\begin{pf}[Index Rules]
    Let $g$ be a primitive root of $p$. By definition, we have that \nmod{g^{I(a)}}{a}{p} and \nmod{g^{I(b)}}{b}{p} \sspace
    \begin{center}
        \nmod{g^{I(ab)}}{ab}{p} \\
        \nmod{ab}{g^{I(a)}g^{I(b)}}{p} \\
        $\implies{}$ \nmod{g^{I(ab)}}{g^{I(a)+I(b)}}{p} \\
        $\implies{}$ \nmod{g^{I(ab)-(I(a)+I(b))}}{1}{p} \\
        $\implies{}$ \nmod{I(ab)-(I(a)+I(b))}{0}{p}
    \end{center} \qed
\end{pf}
\newpage
\paragraph{Example:} Let $p=13,g=2$, compute $12 \cdot 11$ mod 13
\begin{ans}
    Solving \nmod{12\cdot11}{x}{13}
    \begin{center}
        \nmod{I(12\cdot11)}{I(12)+I(11)}{12} \\
        \nmod{I(12)+I(11)}{6+7}{12} \\
        \nmod{6+7}{1}{12}
    \end{center}
    So, we have
    \begin{center}
        \nmod{12\cdot11}{2^{I(12\cdot11)}}{13} \\
        \nmod{12\cdot11}{2}{13}
    \end{center}
\end{ans}

\paragraph{Example:} Let $p=13,g=2$, compute $11^{100}$ mod 13
\begin{ans}
    Solving \nmod{11^{100}}{x}{13}
    \begin{center}
        \nmod{I(11^{100})}{100I(11)}{12} \\
        \nmod{I(11^{100})}{100\cdot7}{12} \\
        \nmod{I(11^{100})}{4}{12}
    \end{center}
    So, we have
    \begin{center}
        \nmod{11^{100}}{2^{I(11^{100})}}{13} \\
        \nmod{11^{100}}{2^4}{13} \\
        \nmod{11^{100}}{3}{13}
    \end{center}
\end{ans}
\paragraph{Example:} Let $p=13,g=2$, solve the linear congruence \nmod{11x}{2}{13}
\begin{ans}
    Take the index of both sides,
    \begin{center}
        \nmod{I(11x)}{I(2)}{12} \\
        \nmod{I(11)+I(x)}{I(2)}{12} \\
        \nmod{7+I(x)}{1}{12} \\
        \nmod{I(x)}{6}{12} \\
        $\implies$ \nmod{x}{12}{13}
    \end{center}
\end{ans}
\paragraph{Remark:} It turns out that it does not matter which primitive root we use, taking the primitive root $g=7$, for example, produces the same set of solutions. It seems like using indices can greatly simplify the computation of linear congruence, however to use such a method, the index table must first be calculated.


\end{document}
