\documentclass[../main.tex]{subfiles}
\begin{document}

\chapter{Pythagorean Triplets}

\section{Integral Pythagorean Triplets}
\paragraph{Question:}
Find all integral Pythagorean triples.

In other words, find all integer triples \(a, b, c\), such that \(a, b, c\) satisfy the Pythagorean formula,
\(c^2 = a^2 + b^2\). We may assume that \(a, b > 0\) since if \(a = 0\) or \(b = 0\) then the solutions are known.

Find all triplets \(a, b, c\) such that $a, b, c \in $ $\N$ and $c^2 = a^2 + b^2$

\paragraph{Examples:}
\begin{center}
    $5^2 = 3^2 + 4^2$ \\
    $13^2 = 5^2 + 12^2$ \\
    $17^2 = 8^2 + 15^2$ \\
    $25^2 = 7^2 + 24^2$ 
\end{center}
Note that, given a Pythagorean triple $a,b,c$, and an integer $d$, then $da,db,dc$ form another Pythagorean triplet.

\begin{defn}
    A Primitive Pythagorean triplet, or $PPT$, is a triple of natural numbers $a, b, c$ so that $c^2=a^2+b^2$ and $gcd(a, b, c) = 1$ since otherwise, you can divide by a common factor and find a smaller triplet.
\end{defn}

We can now rewrite the question into a simpler question, which reduces the question into the essential component of it: Find all primitive Pythagorean triples.

\begin{prop}
    If $(a,b,c)$ form a $PPT$ so that $c^2=a^2+b^2$, then $c$ is odd.
\end{prop}

\begin{pf}
Suppose $c$ is even.
Then $c=2k$ for $k \in \N$


\underline{Case 1:} both $a$ and $b$ are even. \\
Then $gcd(a, b, c) = 2 \neq 1$ thus $a, b, c$ are not $PPT$

\underline{Case 2:} both $a$ and $b$ are odd. \\
Then there exists $x, y \in \Z$ such that $a = 2x + 1$ and $b = 2y + 1$, and since $c^2 = a^2 + b^2$, we have that:

\begin{center}
    $c^2 = (2x + 1)^2 + (2y + 1)^2$ \\
    $4z^2 = 4(x^2 + x y^2 + y) + 2$ \\
    $4(z^2 - x^2 - x - y^2 - y) = 2$ \\
    $2(z^2 - x^2 - x - y^2 - y) = 1$
\end{center}

This is a contradiction, and thus, our first assumption is invalid. Therefore $c$ must be an odd integer. \qed
\end{pf}

So far, we know that if $a, b, c$ is a $PPT$, then $c$ is odd and without loss of generality, we may assume that $a$ is even and $b$ is odd.

By rearranging the Pythagorean formula, we get:

\centerline{$a = \sqrt{(c-b)(c+b)}$}

\begin{prop}
    If $a$ is odd, then $(c-b)$ and $(c+b)$ are squares.
\end{prop}

\begin{pf}
    Let $p$ be a prime divisor of $a$. Then, 
    \begin{center}
        $p^2 \mid a^2$ \\
        $p^2 \mid (c + b)(c - b)$
    \end{center}

    Suppose $p$ divides one of $(c + b)$ and $(c - b)$ but not both, then $(c + b)$ and $(c - b)$ are squares by the \textit{Unique Factorization Theorem.} \\ 
    Therefore it is enough to show that $gcd(c + b, c - b) = 1$.
    
    Let $d = gcd(c + b, c - b)$, then we have the following:
    \begin{center}
        $d \mid c + b$ \\
        $d \mid c - b$ \\
        $\implies d \mid (c + b) + (c - b) = 2c$ \\
        $\implies d \mid (c + b) - (c - b) = 2b$ \\
    \end{center}
\end{pf}
\begin{pf}[Cont.]
    Thus, we have that,
    \begin{center}
        $d \mid gcd(2b, 2c)$ \\
        $d \mid 2gcd(b, c)$ \\
        $d = 1$ or $d = 2$
    \end{center}
    Suppose $d = 2$, we know that,
    \begin{center}
        $d \mid (c-b)$ and $d \mid (c-b)(c+b) = a^2$ \\
        $\implies 2 \mid a^2$ \\
        $\implies 2 \mid a$ \\
        $\implies$ a is even
    \end{center}
    But we assumed that a was odd, so $d$ must be equal to 1. Since $d=gcd(c+b,c-b)$ and $d=1$, $(c-b)$ and $(c+b)$ are coprime, and so are squares by the\textit{Unique Factorization Theorem}. \qed
\end{pf}
Since we have that $(c-b)$ and $(c+b)$ are coprime and squares, we also have that $\exists$ $s, t \in \N$ where $gcd(s,t)=1$ and $c-b = t^2$ and $c+b = s^2$
\begin{center}
    $\implies c = \dfrac{s^2+t^2}{2}$ \\
    $b = \dfrac{s^2-t^2}{2}$ \\
    $a^2 = c^2 - b^2 = s^2 \cdot t^2$ \\
    $\implies a = s \cdot t$
\end{center}
\begin{thm}[Pythagorean Triple Theorem]
    Every primitive Pythagorean triple $(a,b,c)$, where $a,c$ are odd, $b$ is even, can be obtained by $a=s \cdot t$, $b=\dfrac{s^2 - t^2}{2}$, $c=\dfrac{s^2 + t^2}{2}$, where $s,t \in \N, s > t$, and $s,t$ are odd and coprime $(gcd(s,t)=1)$.
\end{thm}
\paragraph{Example: $s = 3$, $t = 1$}
\begin{center}
    $b=\dfrac{3^2-1^2}{2} = 4$ \\
    $c=\dfrac{3^2+1^2}{2} = 5$ \\
    $a=3 \cdot 1 = 3$ \\
    A $PPT$ is $(3, 4, 5)$
\end{center}
It's Important to note that this does not guarantee the result to be a $PPT$ for any $s$ and $t$ satisfying the conditions, however, every $PPT$ will have an $s$ and $t$ decomposition.

\end{document}