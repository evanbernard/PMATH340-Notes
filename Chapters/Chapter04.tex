\documentclass[../main.tex]{subfiles}
\begin{document}

\chapter{Squares Modulo P}
\paragraph{Question:} Given a number $a$, and prime $p$, does the congruence \smod{x}{a}{p} have a solution?

\paragraph{Example:} is 3 congruent to a perfect square modulo 11? ie. does \smod{x}{3}{11} have a solution?
\begin{ans}
    As a naive first attempt, we will just calculate the congruence relation for all unique values of $x$.
    \begin{center}
        \begin{tabular}{c c}
             \smod{1}{1}{11} & \smod{10}{1}{11} \\
             \smod{2}{4}{11} & \smod{9}{4}{11} \\
             \smod{3}{9}{11} & \smod{8}{9}{11} \\
             \smod{4}{5}{11} & \smod{7}{5}{11} \\
             \smod{5}{3}{11} & \smod{6}{3}{11}
        \end{tabular}
    \end{center}
    Clearly, when $x=5,6$, the relation \smod{x}{3}{11} holds.
\end{ans}
It's no coincidence that \smod{1}{10^2}{11} and \smod{2}{9^2}{11} etc. It follows from the fact that we're taking $(p-b)^2$ for some $b$. Expanding that gives $p^2-2pb +b^2$, which is congruent to $b^2$ in mod $p$. Hence, for $p-a=b$, we have that \nmod{a^2}{b^2}{p}.

\begin{defn}[Quadratic Residue and Non-Residue]
    Let $p$ be prime and $a \in \Z$ with $p \nmid a$. If \smod{x}{a}{p} has a solution, then $a$ is called a quadratic residue modulo $p$. We use the notation QR. If \smod{x}{a}{p} does not have a solution, then we call $a$ a quadratic non-residue modulo $p$, and give the notation NR.
\end{defn}

\paragraph{Example:} The quadratic residues in modulo 11 are 1,3,4,5,9 since all of \smod{x}{a}{p} have solutions for $a=1,3,4,5,9$.

It turns out that determining the solution of \smod{x}{a}{p} is very difficult, so we need some observations which will help us in finding the answer.

\begin{prop}
    Let $p$ be an odd prime. Then, there are exactly
    \begin{center}
        $\dfrac{p-1}{2}$ quadratic residues modulo $p$, and \\
        $\dfrac{p-1}{2}$ quadratic non-residues modulo $p$
    \end{center}
\end{prop}
\begin{pf}
    The quadratic residues modulo $p$ are congruent to $1^2,2^2,...,((p-1)/2)^2$. To prove that there are exactly $(p-1)/2$ quadratic residues modulo $p$, it is sufficient to prove that $1^2,2^2,...,((p-1)/2)^2$ are distinct modulo $p$. \sspace
    Suppose that $1 \leq b_2 \leq b_1 \leq \dfrac{p-1}{2}$ and \nmod{b_1^2}{b_2^2}{p}. Then, we have that $p \mid b_1^2 - b_2^2 = (b_1-b_2)(b_1+b_2)$, and since $p$ is prime, either $p \mid (b_1-b_2)$ or $p \mid (b_1+b_2)$. \\
    But $2 \leq b_1+b_2 \leq \dfrac{p-1}{2}+\dfrac{p-1}{2}=p-1<p$, so $p \nmid (b_1+b_2)$ \sspace
    Thus, $p \mid (b_1-b_2)$. But $0 \leq (b_1-b_2) \leq \dfrac{p-1}{2} < p$, so we must have that $b_1-b_2=0$ if $p \mid (b_1-b_2).$ Hence we have that $b_1=b_2$. \sspace
    We have shown that all quadratic residues modulo $p$ are of the form $1^2,2^2,...,((p-1)/2)^2$ are distinct, and thus there are exactly $\dfrac{p-1}{2}$ of them. \qed
\end{pf}

\paragraph{Question:} Suppose $a_1,a_2 \in \Z$. Can we determine whether or not $a_1 \cdot a_2$ is a quadratic residue on the basis of $a_1$ and $a_2$?

\begin{thm}[Quadratic Residue Rules V1]
    Let $p$ be an odd prime. We have the following, where $QR$ denotes an integer which is a quadratic residue modulo $p$, and $NR$ an integer which is a quadratic non-residue.
    \begin{center}
        $QR \cdot QR = QR$ \\
        $QR \cdot NR = NR$ \\
        $NR \cdot NR = QR$
    \end{center}
\end{thm}
\begin{pf}[Quadratic Residue Rules V1]
    We have 3 statements to prove:
    \begin{enumerate}
        \item $QR \cdot QR = QR$
        \item $QR \cdot NR = NR$
        \item $NR \cdot NR = QR$
    \end{enumerate}
    We will begin by proving (1).
    Suppose we have two quadratic residues modulo $p$, namely $a_1,a_2$. By definition, for some integers $b_1,b_2$,
    \begin{center}
        \smod{b_1}{a_1}{p} and \smod{b_2}{a_2}{p} \\
        $\implies$ \nmod{b_1^2b_2^2}{a_1a_2}{p} \\
        $\implies$ \nmod{(b_1b_2)^2}{a_1a_2}{p}
    \end{center}
    Thus, $a_1a_2$ is a QR modulo $p$, as required. \sspace
    Now, lets prove (2).
    Suppose we have a QR and NR $a_1,a_2$ respectively, modulo $p$. Assume that $a_1a_2$ is a QR modulo $p$. Then, we have that,
    \begin{center}
        $\exists b_2 \in \Z$ such that \smod{b_2}{a_1a_2}{p} \\
        But, since $a_1$ is a QR modulo $p$, \\
        $\exists b_1 \in \Z$ such that \smod{b_1}{a_1}{p} \\
        $\implies$ \smod{b_2}{b_1^2a_2}{p} \\
        $\implies$ $b_2^2(b_1^2)^{-1} \equiv b_1^2a_2(b_1^2)^{-1} \equiv a_2$ $(mod$ $p)$
    \end{center}
    Thus, $a_2$ is a QR, which contradicts our assumption. \sspace
    Finally, lets prove (3).
    Let $A$ be the set of all QR's modulo $p$, and $B$ be the set of all NR's modulo $p$. As we've seen, $\abs{A}=\abs{B}=(p-1)/2$. It's also clear that $A \cap B=\emptyset$, and $A \cup B=R_p$. Let $c$ be a NR. Since $cA=\{ca:a\in A\}$, and each element in $A$ is a QR, by (2), $cA$ is a set of NR's, thus $cA \subseteq B$. But, we have that $\abs{A}=\abs{B}$, and $\abs{cA}=\abs{A}$, so we must have that $cA=B$. By a similar argument, we can also get that $cR_p=R_p$.
    \begin{center}
        $A\cup B=R_p$ and $A\cap B=\emptyset$\\
        $B=R_p-A$ \\
        $cB=c(R_p-A)$ \\
        $cB=cR_p-cA$ \\
        $cB=R_p-B$ \\
        $cB=A$
    \end{center}
    So, any NR $c$ times an NR in $B$ is in the set $A$, this is a QR, as required. \qed 
\end{pf}

\begin{defn}[Legendre Symbol]
    Let $p$ be an odd prime, and $gcd(a,p)=1$. The Legendre symbol of $a$ modulo $p$ is given by
    $$\bigg(\frac{a}{p}\bigg):= 
    \begin{cases} 
      1:a\in QR \\
      -1:a \in NR
   \end{cases}$$
\end{defn}

\paragraph{Examples:} $\legendre{7}{11}=-1$, since 7 is an NR modulo 11. $\legendre{9}{11}=1$, since 9 is a QR modulo 11.

\begin{thm}[Quadratic Residue Rules V2]
    For all $a,b \in \Z$ such that $a \not\equiv 0$ $(mod$ $p)$, $b \not\equiv 0$ $(mod$ $p)$, and an odd prime $p$,
    $$\bigg(\frac{ab}{p}\bigg) = \bigg(\frac{a}{p}\bigg) \bigg(\frac{b}{p}\bigg)$$
\end{thm}
\begin{pf}[Quadratic Residue Rules V2]
    
\end{pf}
\end{document}
