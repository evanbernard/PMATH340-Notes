\documentclass[../main.tex]{subfiles}
\begin{document}

\chapter{$x^2-Dy^2=1$}
\section{Square and Triangle Numbers}
\begin{defn}[Square and Triangle Numbers]
    An integer $N$ is called a square number if a square with side length of $n$ can be constructed, such that the number of cubes in the square is equal to $N$. Similarly, $M$ is called a triangle number if a right angle triangle with sides at the right angle equal to $m$ can be constructed such that the number of cubes in the triangle is equal to $M$.
\end{defn}

\paragraph{Example:} Here, we have a square number $N=25$ and a triangle number $M=15$ \\
\setlength{\unitlength}{0.20mm}
\thicklines
%middle is about x=420
\begin{picture}(400,250)
    \put(40,0){\line(1,0){200}}
    \put(40,40){\line(1,0){200}}
    \put(40,80){\line(1,0){200}}
    \put(40,120){\line(1,0){200}}
    \put(40,160){\line(1,0){200}}
    \put(40,200){\line(1,0){200}}
    
    \put(40,0){\line(0,1){200}}
    \put(80,0){\line(0,1){200}}
    \put(120,0){\line(0,1){200}}
    \put(160,0){\line(0,1){200}}
    \put(200,0){\line(0,1){200}}
    \put(240,0){\line(0,1){200}}
    \put(135, -15){$n$}
    \put(245, 95){$n$}
    
    
    \put(600, 0){\line(1,0){200}}
    \put(600, 40){\line(1,0){200}}
    \put(640, 80){\line(1,0){160}}
    \put(680, 120){\line(1,0){120}}
    \put(720, 160){\line(1,0){80}}
    \put(760, 200){\line(1,0){40}}
    
    \put(800, 200){\line(0,-1){200}}
    \put(760, 200){\line(0,-1){200}}
    \put(720, 160){\line(0,-1){160}}
    \put(680, 120){\line(0,-1){120}}
    \put(640, 80){\line(0,-1){80}}
    \put(600, 40){\line(0,-1){40}}
    
    \put(805, 95){$m$}
    \put(695,-15){$m$}
\end{picture}
\newpage
\paragraph{Question:} Find all the square numbers that are also triangular numbers. \sspace
We need to solve $n^2=\dfrac{m(m+1)}{2}$ in $\N$. 
\begin{center}
   $n^2=\dfrac{m(m+1)}{2}$ \\
   $8n^2=4m^2+4m$ \\
   $2(2n)^2=(2m+1)^2-1$
\end{center}
Setting $x=2m+1$ and $y=2n$, we have 
\begin{center}
    $2y^2=x^2-1$ \\
    $x^2-2y^2=1$
\end{center}
So, if there is a solution $x=2m+1$ for some $m$, then $y$ must also be even. Thus, let $m=\tfrac{x-1}{2}$ and $n=\tfrac{y}{2}$. So, we need to solve $x^2-2y^2=1$ in $\Z$ to find all the numbers which are both square and triangle. We can see the first few solutions by inspection:
\begin{center}
    $m=1$, $n=1$ $\implies x=3$, $y=2$ \\
    $m=8$, $n=6$ $\implies x=17$, $y=12$ \\
    $m=49$, $n=35$ $\implies x=99$, $y=70$
\end{center}
Now to try to figure out the pattern, we will consider the much simpler equation $x^2-y^2=1$
\begin{center}
    $x^2-y^2=1$ \\
    $(x-y)(x+y)=1$ \\
    $x=\pm 1$, $y=0$
\end{center}
So, it's clear that the difficulty of $x^2-2y^2=1$ arises from the fact that $2$ is a non-unit coefficient of $y$ in $\Z[x,y]$, so we cannot factor it in $\Z[x,y]$. What if we allow ourselves to use $\sqrt{2}$? Then, we may factor 
\begin{center}
    $x^2-2y^2=1$ \\
    $(x-\sqrt{2}y)(x+\sqrt{2}y)=1$
\end{center}
We can find the first solution manually, and we find that $x=3,y=2$ is a solution. So, we have $(3-2\sqrt{2})(3+2\sqrt{2})=1$. Notice that we may take this to the $k^{th}$ power and obtain another solution. Taking it to the second power, we get
\begin{center}
    $(3-2\sqrt{2})^2(3+2\sqrt{2})^2=1^2$ \\
    $(17-12\sqrt{2})(17+12\sqrt{2})=1$ \\
    $17^2-2(12)^2=1$ \\
    $\implies x=17, y=12$
\end{center}
So, it's clear that we may find infinitely many solutions by taking $(3-2\sqrt{2})(3+2\sqrt{2})=1$ to the power of $k$, for $k \in \N$. This can be proven explicitly using the binomial theorem, but will be omitted here. The natural question to ask is whether or not doing this for all $k$ will find every solution, and the answer is yes!

\begin{thm}
    \begin{enumerate}
        \item Every positive integer solution to the equation $x^2-2y^2=1$ is obtained by raising $3+2\sqrt{2}$ to different powers. That is, the solutions $(x_k,y_k)$ can be found from $x_k+y_k\sqrt{2}=(3+2\sqrt{2})^k$, $k \in \N$
        \item Every square-triangular number $n^2=\tfrac{m(m+1)}{2}$ is given by $m=\tfrac{x_k-1}{2}$, and $n=\tfrac{y_k}{2}$
    \end{enumerate}
\end{thm}
\begin{pf}
    It's easy to see that raising $(3+2\sqrt{2})$ to some $k^{th}$ power will yield a solution to $x_k^2-2y_k^2=1$, and we've already shown that every square-triangular number is given by $m=\tfrac{x_k-1}{2}$ and $n=\tfrac{y_k}{2}$, so it remains to show that every solution $(x_k,y_k)$ to $x^2-2y^2=1$ is obtained by taking $(3+2\sqrt{2})$ to some power $k$. i.e, for every solution $(u,v)$, there exists some $k \in \N$ such that $u+v\sqrt{2}=(3+2\sqrt{2})^k$ \sspace
    Let $(u,v)$ be a solution, so $u^2-2v^2=1$. \\
    \underline{Case 1:} $u=1$, There are no positive solutions \\
    \underline{Case 2:} $u=2$, There are no solutions since $u^2$ is even \\
    \underline{Case 3:} $u=3$, we have a solution $(3,2)$, and $k=1$ gives $(u+\sqrt{2}v)=(3+\sqrt{2}2)^k$ 
    So, we may assume that $u > 3$. \sspace
    Claim: Let $u,v$ be a solution to $x^2-2y^2=1$ such that $u>3$. Then, we can find two positive integers $t,s$ where $s < u$ such that $u+v\sqrt{2}=(3+2\sqrt{2})(s+t\sqrt{2})$. Note that this is a descent argument. \sspace
    Proof of Claim: We can always write $u+v\sqrt{2}=(3+2\sqrt{2})(s+t\sqrt{2})$. In fact, multiplying by $(3-2\sqrt{2})$, we get
    \begin{center}
        $(3-2\sqrt{2})(u+v\sqrt{2})=(3-2\sqrt{2})(3+2\sqrt{2})(s+t\sqrt{2})$ \\
        $=(s+t\sqrt{2})$, thus, \\
        $(3u-4v)+(3v-2u)\sqrt{2}=s+t\sqrt{2}$
    \end{center}
    So, we can set $$s=3u-4v\text{ and }t=3v-2u$$ to get that $u+v\sqrt{2}=(3+2\sqrt{2})(s+t\sqrt{2})$, as required. Moreover, $s^2-2t^2=1$, which can be seen directly. We now need to show the following
    \begin{enumerate}
        \item $s,t > 0$
        \item $s < u$
    \end{enumerate}
\end{pf}
\begin{pf}[Cont.]
    For 1., note that $u^2-2v^2=1$, so $u^2=1+2v^2$, and $u > v\sqrt{2}$. Thus, $\tfrac{u}{v}>\sqrt{2}$, but we have that
    \begin{center}
        $s=3u-4v$ \\
        $s=3v(\tfrac{u}{v}-\tfrac{4}{3}) > 3v(1.414...-1.333...)>0$
    \end{center}
    With a very similar argument, you can see that $t >0$, thus we have that $t,s$ are positive integers, as required. \sspace
    For 2., note that 
    \begin{align*}
        u+\sqrt{2}v=(3+2\sqrt{2})(s+t\sqrt{2}) \\
        =(3s+4t)+(2s+3t)\sqrt{2} \\
        \implies u=3s+4t, v=2s+3t
    \end{align*}
    Notice that $u > s$ and $v>t$ since $s,t > 0$, so $s < u$ as required.
\end{pf}

\section{Pell Equation}
\begin{defn}[Pell Equation]
    A Pell equation is an equation of the form $x^2-Dy^2=1$, where $D \in \N$ is not a perfect square.
\end{defn}
Notice that if $D$ were a perfect square, it would be easy to factor the left hand side into a product of linear polynomials, and subsequently find integer solutions.
\paragraph{Example:} $D=2$ gives the Pell equation $x^2-2y^2=1$, which as we've just shown, has the set of integer solutions obtained by raising $3+2\sqrt{2}$ to all positive integers. \\
\begin{thm}
    If we have some solution $(x_0,y_0)$ to a Pell equation $x^2-Dy^2=1$ for some non perfect square $D$, then we have infinitely many solutions to that equation.
\end{thm}
\begin{pf}
    We know that $(x_0-\sqrt{D}y_0)(x_0+\sqrt{D}y_0)=1$, so we can raise the equation to the $k^{th}$ power for some $k \in \N$, and we get
    \begin{center}
        $1=(x_0-\sqrt{D}y_0)(x_0+\sqrt{D}y_0)$ \\
        $1^k=1=(x_0-\sqrt{D}y_0)^k(x_0+\sqrt{D}y_0)^k$ \\
        $=(x_k-\sqrt{D}y_k)(x_k+\sqrt{D}y_k)$ \\
        $=x_k^2-Dy_k^2$
    \end{center}
    Thus, $(x_k,y_k)$ is a solution to $x^2-Dy^2=1$, showing that if we have one solution to a Pell equation, then we have infinitely many solutions.
\end{pf}
\paragraph{Example:} Let $D=313$. The smallest positive integer solution to $x^2-313y^2=1$ is \\ \centerline{$x=32188120829134849$, $y=1819380158564160$}
\paragraph{Observations:} Intuitively, if we have very large $x,y \in \N$, then $x-\sqrt{D}y$ is very small. So if we re-arrange the Pell equation, we see that \\ \centerline{$\dfrac{1}{x+\sqrt{D}y}$ is very small} Multiplying by $\tfrac{1}{y}$, we see that \\ \centerline{$\dfrac{x}{y}-\sqrt{D}=\dfrac{1}{x+\sqrt{D}y}\dfrac{1}{y}$ is even smaller} So, when $x,y$ are very large solutions to $x^2-Dy^2=1$, then $\tfrac{x}{y}$ is a good approximation to $\sqrt{D}$. \\ Notice that since $x,y$ are positive integers, we get that $$\abs{\dfrac{x}{y}-\sqrt{D}} \leq \dfrac{1}{\sqrt{D}y^2}$$
\paragraph{Observation:} For any $y \in \N$, if we take $x$ to be the integer closest to $\sqrt{D}y$, then $$\abs{x-\sqrt{D}y} \leq \dfrac{1}{2} \implies \abs{\dfrac{x}{y}-\sqrt{D}} \leq \dfrac{1}{2y}$$


\begin{thm}[The Pigeonhole Principle]
    \underline{Finite Version:} If we have more pigeons than pigeonholes, then at least one pigeonhole has two pigeons. \\
    \underline{Infinite Version:} If we have finitely many pigeon holes and infinitely many pigeons, then at least one pigeonhole has infinitely many pigeons.
\end{thm}

\begin{thm}[Dirichlet Diophantine Approximation Theorem]
    Suppose that $D \in \N$ and $D$ is not a perfect square. Then, there are infinitely many pairs of $(x,y)$ such that $$\abs{\dfrac{x}{y}-\sqrt{D}} \leq \dfrac{1}{y^2}$$ 
\end{thm}
\paragraph{Remark:} This actually holds for any irrational number $d$, but we insist here that $d=\sqrt{D}$ for some non perfect square $D$

\paragraph{Claim:} Let $Y \in \N$. Then, there exists a pair $(x,y)$, where $x,y \in \N$ such that $x-y\sqrt{D} \leq \tfrac{1}{Y}$, with $1 \leq y \leq Y$

\begin{pf}[of Claim]
    We set up $Y$ pigeon holes. Start with the interval $[0,1]$, and divide it into $Y$ sub interval, so we have the first interval $[0, \tfrac{1}{Y}]$, the second interval $[\tfrac{1}{Y}, \tfrac{2}{Y}]$, and the last interval $[\tfrac{Y-1}{Y}, 1]$. We will now argue that we must have some $(x,y)$ in the same interval, which will imply the result we are looking for. \sspace
    We may write 
    \begin{center}
        $0\sqrt{D} = N_0+F_0 = \floor{0\sqrt{D}}+F_0$ so $0 \leq F_0 \leq 1$ \\
        $1\sqrt{D} = N_1+F_1$ \\
        $t\sqrt{D} = N_t+F_t$
    \end{center}
    Notice that we may also write the fraction parts $F_t$ as
    \begin{center}
        $F_t = t\sqrt{D}-N_t$ \\
        $F_t = t\sqrt{D}-\floor{t\sqrt{D}}$
    \end{center}
    So, for $0 \leq t \leq Y$, we can look at the set of fraction parts, $\{F_0,F_1,...,F_Y\}$, noticing that we have $(Y+1)$ fraction parts, and all of the fraction parts are in the range $0 \leq F_i \leq Y$, and we only have $Y$ intervals. Thus, by the finite version of the pigeonhole argument, we must have at least 2 elements in one interval. Namely, there exists $0 \leq m < n \leq Y$ such that $F_n,F_m \in [\tfrac{s-1}{Y}, \tfrac{s}{Y}]$. Thus, $$\abs{F_m-F_n} < \tfrac{1}{Y}$$ So, expanding the fraction parts, we get that we can find a solution 
    \begin{center}
        $x=(N_n-N_m)$ \\
        $y=(n-m)$
    \end{center}
    such that $x-y\sqrt{D} < \tfrac{1}{Y}$, as required. \qed
\end{pf}
\begin{pf}[DDAT]
    Using induction, the base case is $n=1$, and we we are done. \sspace
    Assume that we have found some solution $(x_i,y_i)$, $1 \leq i \leq n$ such that $\abs{x_i-y_i\sqrt{D}} \leq \tfrac{1}{y_i}$. \sspace
    Let $Y_{n+1}$ be the positive integer such that $\tfrac{1}{Y_{n+1}}$ is smaller than any $\abs{x-y\sqrt{D}}$, where $x,y \in \N$, $1 \leq y \leq max\{y_1,y_2,...,y_n\}$. By the claim, there exists $x_{n+1},y_{n+1}$ such that $\abs{x_{n+1}-y_{n+1}} < \tfrac{1}{Y_{n+1}}$, where $y_{n+1} \leq Y_{n+1}$. By the choice of $Y_{n+1}$, we know that $y_{n+1} > max\{y_1,...,y_n\}$. In particular, $y_{n+1} \neq y_i$ for all $0 \leq i \leq n$. So, $(x_{n+1}, y_{n+1})$ is different than $(x_i,y_i)$ for all $1 \leq i \leq n$. \sspace
    Thus, the statement is true for $n+1$. So, by POSI, the statement is true for all $n$. \qed
\end{pf}

By DDAT, we have infinitely many pairs $(x,y)$ such that $$\abs{x-y\sqrt{D}} \leq \dfrac{1}{y}$$ If we multiply both sides by $\abs{x+y\sqrt{D}}$, we get $$\abs{x^2-Dy^2} \leq \dfrac{x+y\sqrt{D}}{y} = \dfrac{x}{y} + \sqrt{D}$$ Moreover, we know that $$\abs{\tfrac{x}{y}-\sqrt{D}} \leq \tfrac{1}{y^2}$$ $$\tfrac{x}{y} \leq \sqrt{D}+\tfrac{1}{y^2} \leq 2\sqrt{D}$$ To sum up, we have $$(x^2-Dy^2) \leq \dfrac{x}{y}+\sqrt{D} \leq 3\sqrt{D}$$

Remember, we want to solve $x^2-\sqrt{D}y^2=1$, so we need to work our way down from being less than or equal to $3\sqrt{D}$ to being equal to 1. To start, we'll use a pigeonhole argument. Consider the set of pigeons $$S_1=\{x^2-Dy^2:x,y \in \N, \abs{x^2-Dy^2} \leq 3\sqrt{D}\}$$ Clearly, each element in $S_1$ is bounded by $3\sqrt{D}$. Now consider $2A+1$ pigeonholes, where $A=\ceil{3\sqrt{D}}$, with the set of pigeonholes being of the form $$S_2=\{-A,-A+1,...,0,1,...,A\}$$ So, by DDAT, we have infinitely many elements in the set $S_1$, and only a finite number of elements in the set $S_2$, so by the pigeonhole argument, there must be some element in $S_2$ which has an infinite number of pigeons in it. More precisely, there exists some $M \in \Z$, $-A \leq M \leq A$, $M \neq 0$ such that there are infinitely many positive integer pairs $(x,y)$ which satisfy $x^2-Dy^2=M$. \sspace

Before continuing with the proof, lets look at a concrete example.
\paragraph{Example:} Consider $D=13$ and $M=4$. Find some $x,y \in \N$ such that $x^2-Dy^2=1$.
\begin{ans}
    We have that $x^2-13y^2=4$ has solutions $(x,y)=(11,3),(119,33),(14159,3927)$. Now we would like to find a solution to $x^2-13y^2=1$. The idea to do this is to divide one "bigger" solution by a different solution. Considering the solutions $(11,3),(119,33)$, we get 
    \begin{center}
        $11^2-13(3^2)=4$ \\
        $(11-3\sqrt{13})(11+3\sqrt{13})$ \\
        $(119-33\sqrt{13})(119+33\sqrt{13})$
    \end{center}
    Notice that we have $$\dfrac{(119-33\sqrt{13})(119+33\sqrt{13})}{(11-3\sqrt{13})(11+3\sqrt{13})}=\dfrac{4}{4}=1$$
    The division of $x_1-y_1\sqrt{D}$ by $x_2-y_2\sqrt{D}$ clearly gives an expression of the form $x_3-y_3\sqrt{D}$, and similarly, the division of the conjugates will give an expression of the form $x_3+y_3\sqrt{D}$. Thus, performing the division of one solution on another will give a solution to $x^2-Dy^2=1$. The only issue is, we cannot guarantee that it will produce integer values for $x,y$. For example, let's divide $119-33^2\sqrt{13}$ by $11-3\sqrt{13}$. 
    \begin{center}
        $\dfrac{119-33\sqrt{13}}{11-3\sqrt{13}}=\dfrac{119-33\sqrt{13}}{11-3\sqrt{13}}\dfrac{11+3\sqrt{13}}{11+3\sqrt{13}}$ \\
        $=\dfrac{22-6\sqrt{13}}{4}$
    \end{center}
    We can check that $x,y=(\tfrac{22}{4},\tfrac{6}{4})$ is indeed a solution to $x^2-13y^2=1$, but $x,y$ are not integers. Let's try dividing the solution $(14159,3927)$ by the solution $(11,3)$. The details will be omitted here, but one can check for themselves that the division yields $649-180\sqrt{13}$, thus we have an integer solution $(649,180)$ to $x^2-13y^2=1$. Why did the division of this solution work? Well, when multiplying the numerator by the conjugate of the denominator, if we have that $c^2-Dy^2=M$, we have the general form
    \begin{center}
        $\dfrac{a-b\sqrt{D}}{c-d\sqrt{D}}\cdot\dfrac{c+d\sqrt{D}}{c+d\sqrt{D}}=\dfrac{ac-bdD+(ad-cb)\sqrt{D}}{M}$
    \end{center}
    So, to get an integer solution, we must have that $M \mid (ac-bdD)$ and $M \mid (ad-cb)$ Thus, we get the following
    \begin{center}
        \nmod{c}{a}{M} \\
        \nmod{d}{b}{M}
    \end{center}
    In our example we see that indeed, \nmod{11}{14159}{4} and \nmod{3}{3927}{4}, which is why we yield an integer solution of $(649,180)$ to $x^2-13y^2=1$ by the division.
\end{ans}
\begin{thm}
    Let $D \in \N$, where $D$ is not a perfect square. Then, the pell equation $x^2-Dy^2=1$ has a positive integer solution.
\end{thm}
\begin{pf}
    By our previous work, there exists some $M \in \Z$, $M \neq 0$ such that there are infinitely many positive integer pairs $(x,y)$ which satisfy $x^2-Dy^2=M$. So, consider the solution 
    \begin{center}
        \nmod{x_i}{x_j}{M} \\
        \nmod{y_i}{y_j}{M}
    \end{center}
    Most of the proof will be omitted here, but the key is to manipulate the fraction $$\dfrac{x_i-y_i\sqrt{D}}{x_j-y_j\sqrt{D}}$$ by multiplying by the conjugate, and replacing $x_i$ with $x_j$ in modulo $M$, to see that the quotient of $x_i-y_i\sqrt{D}$ on $x_j-y_j\sqrt{D}$ has integer values for $x,y$, and thus we can find a positive integer solution to the Pell equation. \qed
\end{pf}
\paragraph{Remark:} We've shown that given the Pell equation $x^2-Dy^2=1$, we can find a positive integer solution, and thus can find infinitely many solutions by taking $(x_0+y_0\sqrt{D})$ to the $k^{th}$ power, where $(x_0,y_0)$ is a solution. The natural thing to wonder is the general form of any solution to the equation, and it turns out that any positive integer solution to the equation can be obtained by taking the 'smallest' positive integer solution $(x,y)$ to some positive power.

\begin{thm}[Pell Equation Theorem]
    Let $D \in \N$, such that $D$ is not a perfect square. Then, the pell equation $x^2-Dy^2=1$ always has a positive integer solution, and if we let $(x_1,y_1)$ be the smallest positive integer solution in terms of $x_1$, then the set of all positive integer solutions to $x^2-Dy^2=1$ is $$\{(x_k,y_k):x_k+y_k\sqrt{D}=(x_1+y_1\sqrt{D})^k, k \in \N\}$$
\end{thm}
\begin{pf}
    We've already shown that the equation has a positive integer solution $(x_1,y_1)$, and we can find infinitely many other positive integer solutions generated by $(x_1+y_1\sqrt{D})$, so it remains to show that all positive integer solutions to the equation can be expressed as a power of $x_1+y_1\sqrt{D}$. We do so by a descent argument. \sspace
    \underline{Claim:} Let $(u,v)$ be a positive integer solution to $x^2-Dy^2=1$, where $u \geq x_1$. Then we can find two positive integers $s,t \in \Z$ such that $s < u$, and $$u+v\sqrt{D}=(x_1+y_1\sqrt{D})(s+t\sqrt{D})$$
\end{pf}
\begin{pf}[cont.]
    Notice that if we can prove the claim, then $(s,t)$ is also a positive integer solution to $x^2-Dy^2=1$, with $s < x_1$, so we can therefore find this new, smaller solution inductively to get the smallest positive integer solution, which shows that all solutions are powers of $(x_1+y_1\sqrt{D})$. \sspace
    We can always write $$u+v\sqrt{D}=(x_1+y_1\sqrt{D})(s+t\sqrt{D})$$ Now, multiplying both sides by $x_1-y_1\sqrt{D}$, 
    \begin{center}
        $(x_1-y_1\sqrt{D})(u+v\sqrt{D})=(x_1^2-Dy^2)(s+t\sqrt{D})$ \\
        $=s+t\sqrt{D}$ thus, \\
        $s=ux_1-vy_1D \in \Z$ \\
        $t=x_1v-uy_1 \in \Z$ 
    \end{center}
    Now we just need to show that both $s$ and $t$ are positive. This will be omitted, but it's easy to see this, since we have explicit expressions for $s$ and $t$. So, $s,t$ are positive integer solutions smaller than the previous solution, and thus we can express any solution as the power of the smallest integer solution, as required. \qed
\end{pf}
Due to COVID-19, we no longer have classes and so the course concludes here. The content covered in this document is up to assignment 8.
\end{document}
